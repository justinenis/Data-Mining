% Options for packages loaded elsewhere
\PassOptionsToPackage{unicode}{hyperref}
\PassOptionsToPackage{hyphens}{url}
%
\documentclass[
]{article}
\usepackage{lmodern}
\usepackage{amssymb,amsmath}
\usepackage{ifxetex,ifluatex}
\ifnum 0\ifxetex 1\fi\ifluatex 1\fi=0 % if pdftex
  \usepackage[T1]{fontenc}
  \usepackage[utf8]{inputenc}
  \usepackage{textcomp} % provide euro and other symbols
\else % if luatex or xetex
  \usepackage{unicode-math}
  \defaultfontfeatures{Scale=MatchLowercase}
  \defaultfontfeatures[\rmfamily]{Ligatures=TeX,Scale=1}
\fi
% Use upquote if available, for straight quotes in verbatim environments
\IfFileExists{upquote.sty}{\usepackage{upquote}}{}
\IfFileExists{microtype.sty}{% use microtype if available
  \usepackage[]{microtype}
  \UseMicrotypeSet[protrusion]{basicmath} % disable protrusion for tt fonts
}{}
\makeatletter
\@ifundefined{KOMAClassName}{% if non-KOMA class
  \IfFileExists{parskip.sty}{%
    \usepackage{parskip}
  }{% else
    \setlength{\parindent}{0pt}
    \setlength{\parskip}{6pt plus 2pt minus 1pt}}
}{% if KOMA class
  \KOMAoptions{parskip=half}}
\makeatother
\usepackage{xcolor}
\IfFileExists{xurl.sty}{\usepackage{xurl}}{} % add URL line breaks if available
\IfFileExists{bookmark.sty}{\usepackage{bookmark}}{\usepackage{hyperref}}
\hypersetup{
  pdftitle={HW 1},
  hidelinks,
  pdfcreator={LaTeX via pandoc}}
\urlstyle{same} % disable monospaced font for URLs
\usepackage[margin=1in]{geometry}
\usepackage{color}
\usepackage{fancyvrb}
\newcommand{\VerbBar}{|}
\newcommand{\VERB}{\Verb[commandchars=\\\{\}]}
\DefineVerbatimEnvironment{Highlighting}{Verbatim}{commandchars=\\\{\}}
% Add ',fontsize=\small' for more characters per line
\usepackage{framed}
\definecolor{shadecolor}{RGB}{248,248,248}
\newenvironment{Shaded}{\begin{snugshade}}{\end{snugshade}}
\newcommand{\AlertTok}[1]{\textcolor[rgb]{0.94,0.16,0.16}{#1}}
\newcommand{\AnnotationTok}[1]{\textcolor[rgb]{0.56,0.35,0.01}{\textbf{\textit{#1}}}}
\newcommand{\AttributeTok}[1]{\textcolor[rgb]{0.77,0.63,0.00}{#1}}
\newcommand{\BaseNTok}[1]{\textcolor[rgb]{0.00,0.00,0.81}{#1}}
\newcommand{\BuiltInTok}[1]{#1}
\newcommand{\CharTok}[1]{\textcolor[rgb]{0.31,0.60,0.02}{#1}}
\newcommand{\CommentTok}[1]{\textcolor[rgb]{0.56,0.35,0.01}{\textit{#1}}}
\newcommand{\CommentVarTok}[1]{\textcolor[rgb]{0.56,0.35,0.01}{\textbf{\textit{#1}}}}
\newcommand{\ConstantTok}[1]{\textcolor[rgb]{0.00,0.00,0.00}{#1}}
\newcommand{\ControlFlowTok}[1]{\textcolor[rgb]{0.13,0.29,0.53}{\textbf{#1}}}
\newcommand{\DataTypeTok}[1]{\textcolor[rgb]{0.13,0.29,0.53}{#1}}
\newcommand{\DecValTok}[1]{\textcolor[rgb]{0.00,0.00,0.81}{#1}}
\newcommand{\DocumentationTok}[1]{\textcolor[rgb]{0.56,0.35,0.01}{\textbf{\textit{#1}}}}
\newcommand{\ErrorTok}[1]{\textcolor[rgb]{0.64,0.00,0.00}{\textbf{#1}}}
\newcommand{\ExtensionTok}[1]{#1}
\newcommand{\FloatTok}[1]{\textcolor[rgb]{0.00,0.00,0.81}{#1}}
\newcommand{\FunctionTok}[1]{\textcolor[rgb]{0.00,0.00,0.00}{#1}}
\newcommand{\ImportTok}[1]{#1}
\newcommand{\InformationTok}[1]{\textcolor[rgb]{0.56,0.35,0.01}{\textbf{\textit{#1}}}}
\newcommand{\KeywordTok}[1]{\textcolor[rgb]{0.13,0.29,0.53}{\textbf{#1}}}
\newcommand{\NormalTok}[1]{#1}
\newcommand{\OperatorTok}[1]{\textcolor[rgb]{0.81,0.36,0.00}{\textbf{#1}}}
\newcommand{\OtherTok}[1]{\textcolor[rgb]{0.56,0.35,0.01}{#1}}
\newcommand{\PreprocessorTok}[1]{\textcolor[rgb]{0.56,0.35,0.01}{\textit{#1}}}
\newcommand{\RegionMarkerTok}[1]{#1}
\newcommand{\SpecialCharTok}[1]{\textcolor[rgb]{0.00,0.00,0.00}{#1}}
\newcommand{\SpecialStringTok}[1]{\textcolor[rgb]{0.31,0.60,0.02}{#1}}
\newcommand{\StringTok}[1]{\textcolor[rgb]{0.31,0.60,0.02}{#1}}
\newcommand{\VariableTok}[1]{\textcolor[rgb]{0.00,0.00,0.00}{#1}}
\newcommand{\VerbatimStringTok}[1]{\textcolor[rgb]{0.31,0.60,0.02}{#1}}
\newcommand{\WarningTok}[1]{\textcolor[rgb]{0.56,0.35,0.01}{\textbf{\textit{#1}}}}
\usepackage{graphicx,grffile}
\makeatletter
\def\maxwidth{\ifdim\Gin@nat@width>\linewidth\linewidth\else\Gin@nat@width\fi}
\def\maxheight{\ifdim\Gin@nat@height>\textheight\textheight\else\Gin@nat@height\fi}
\makeatother
% Scale images if necessary, so that they will not overflow the page
% margins by default, and it is still possible to overwrite the defaults
% using explicit options in \includegraphics[width, height, ...]{}
\setkeys{Gin}{width=\maxwidth,height=\maxheight,keepaspectratio}
% Set default figure placement to htbp
\makeatletter
\def\fps@figure{htbp}
\makeatother
\setlength{\emergencystretch}{3em} % prevent overfull lines
\providecommand{\tightlist}{%
  \setlength{\itemsep}{0pt}\setlength{\parskip}{0pt}}
\setcounter{secnumdepth}{-\maxdimen} % remove section numbering

\title{HW 1}
\author{}
\date{\vspace{-2.5em}}

\begin{document}
\maketitle

\hypertarget{r-markdown}{%
\subsection{R Markdown}\label{r-markdown}}

This is an R Markdown document. Markdown is a simple formatting syntax
for authoring HTML, PDF, and MS Word documents. For more details on
using R Markdown see \url{http://rmarkdown.rstudio.com}.

When you click the \textbf{Knit} button a document will be generated
that includes both content as well as the output of any embedded R code
chunks within the document. You can embed an R code chunk like this:

\begin{Shaded}
\begin{Highlighting}[]
\KeywordTok{summary}\NormalTok{(cars)}
\end{Highlighting}
\end{Shaded}

\begin{verbatim}
##      speed           dist       
##  Min.   : 4.0   Min.   :  2.00  
##  1st Qu.:12.0   1st Qu.: 26.00  
##  Median :15.0   Median : 36.00  
##  Mean   :15.4   Mean   : 42.98  
##  3rd Qu.:19.0   3rd Qu.: 56.00  
##  Max.   :25.0   Max.   :120.00
\end{verbatim}

\hypertarget{including-plots}{%
\subsection{Including Plots}\label{including-plots}}

You can also embed plots, for example:

\includegraphics{HW1_files/figure-latex/pressure-1.pdf}

Note that the \texttt{echo\ =\ FALSE} parameter was added to the code
chunk to prevent printing of the R code that generated the plot.

library(tidyverse) library(ggplot2) library(rsample) library(caret)
library(modelr) library(parallel) library(foreach)

gas\_prices =
read.csv(`/Users/franklinstudent/Desktop/GitHub/ECO395M/data/GasPrices.csv')

\#1A ggplot(data=gas\_prices) + geom\_boxplot(aes(x=Competitors,
y=Price, fill = Competitors))

\hypertarget{in-this-boxplot-we-are-assesing-whether-the-lack-of-direct-competition-in}{%
\section{In this boxplot, we are assesing whether the lack of direct
competition
in}\label{in-this-boxplot-we-are-assesing-whether-the-lack-of-direct-competition-in}}

\hypertarget{sight-results-in-higher-gas-prices.-the-boxplot-is-separated-into-two-categories}{%
\section{sight results in higher gas prices. The boxplot is separated
into two
categories,}\label{sight-results-in-higher-gas-prices.-the-boxplot-is-separated-into-two-categories}}

\hypertarget{yes-or-no-on-whether-a-competitor-is-within-sight-of-the-observed-gas-station.}{%
\section{yes or no, on whether a competitor is within sight of the
observed gas
station.}\label{yes-or-no-on-whether-a-competitor-is-within-sight-of-the-observed-gas-station.}}

\hypertarget{i-also-added-color-to-better-differentiate-between-both-categories.}{%
\section{I also added color to better differentiate between both
categories.}\label{i-also-added-color-to-better-differentiate-between-both-categories.}}

\hypertarget{when-interpreting-the-boxplot-it-becomes-clear-that-the-lack-of-direct}{%
\section{When interpreting the boxplot, it becomes clear that the lack
of
direct}\label{when-interpreting-the-boxplot-it-becomes-clear-that-the-lack-of-direct}}

\hypertarget{competition-in-sight-results-in-higher-gas-prices.-the-median-price-of-gas-stations}{%
\section{competition in sight results in higher gas prices. The median
price of gas
stations}\label{competition-in-sight-results-in-higher-gas-prices.-the-median-price-of-gas-stations}}

\hypertarget{with-no-competitors-in-sight-is-approximately-1.89gal-while-in-contrast-the}{%
\section{with no competitors in sight is approximately \$1.89/gal, while
in contrast,
the}\label{with-no-competitors-in-sight-is-approximately-1.89gal-while-in-contrast-the}}

\hypertarget{median-price-of-gas-stations-with-competitors-in-sight-is-1.85gal.-moreover}{%
\section{median price of gas stations with competitors in sight is
\$1.85/gal.
Moreover,}\label{median-price-of-gas-stations-with-competitors-in-sight-is-1.85gal.-moreover}}

\hypertarget{gas-stations-with-no-direct-competitors-in-sight-have-a-larger-range-of-prices}{%
\section{gas stations with no direct competitors in sight have a larger
range of
prices,}\label{gas-stations-with-no-direct-competitors-in-sight-have-a-larger-range-of-prices}}

\hypertarget{relative-to-gas-stations-with-direct-competitors-in-sight.}{%
\section{relative to gas stations with direct competitors in
sight.}\label{relative-to-gas-stations-with-direct-competitors-in-sight.}}

\#1B ggplot(data = gas\_prices) + geom\_point(mapping = aes(x = Income,
y = Price))

\hypertarget{in-this-scatterplot-we-are-determining-whether-more-affluent-areas-experience}{%
\section{In this scatterplot, we are determining whether more affluent
areas
experience}\label{in-this-scatterplot-we-are-determining-whether-more-affluent-areas-experience}}

\hypertarget{higher-gas-prices.-interpreting-the-data-there-appears-to-be-a-positive}{%
\section{higher gas prices. Interpreting the data, there appears to be a
positive}\label{higher-gas-prices.-interpreting-the-data-there-appears-to-be-a-positive}}

\hypertarget{association-between-the-variables-income-and-price-indicating-yes-more-affluent}{%
\section{association between the variables Income and Price, indicating
yes, more
affluent}\label{association-between-the-variables-income-and-price-indicating-yes-more-affluent}}

\hypertarget{areas-experience-higher-gas-prices.-although-it-should-be-noted-that-there-are}{%
\section{areas experience higher gas prices. Although, it should be
noted that there
are}\label{areas-experience-higher-gas-prices.-although-it-should-be-noted-that-there-are}}

\hypertarget{several-outliers.}{%
\section{several outliers.}\label{several-outliers.}}

\#1C ggplot(data = gas\_prices) + geom\_col(mapping = aes(x = Brand, y =
Price, fill = Brand))

\hypertarget{in-creating-this-barplot-we-are-determining-if-shell-charges-more-for-gasoline}{%
\section{In creating this barplot, we are determining if Shell charges
more for
gasoline}\label{in-creating-this-barplot-we-are-determining-if-shell-charges-more-for-gasoline}}

\hypertarget{relative-to-other-brands.-upon-observing-the-barplot-it-becomes-clear-that-shell}{%
\section{relative to other brands. Upon observing the barplot, it
becomes clear that
Shell}\label{relative-to-other-brands.-upon-observing-the-barplot-it-becomes-clear-that-shell}}

\hypertarget{charges-more-for-gas-than-either-chevron-texaco-or-exxonmobil-but-charges-less}{%
\section{charges more for gas than either Chevron-Texaco or ExxonMobil,
but charges
less}\label{charges-more-for-gas-than-either-chevron-texaco-or-exxonmobil-but-charges-less}}

\hypertarget{than-other.}{%
\section{than Other.}\label{than-other.}}

\#1D ggplot(data = gas\_prices) + geom\_histogram(aes(x = Price,
after\_stat(density)), binwidth = 0.05) +
facet\_wrap(\textasciitilde Stoplight)

\hypertarget{in-this-set-of-data-we-created-a-faceted-histogram-to-assist-in-determing-if}{%
\section{In this set of data, we created a faceted histogram to assist
in determing
if}\label{in-this-set-of-data-we-created-a-faceted-histogram-to-assist-in-determing-if}}

\hypertarget{whether-the-precence-of-stoplights-near-gas-stations-increases-gas-prices.}{%
\section{whether the precence of stoplights near gas stations increases
gas
prices.}\label{whether-the-precence-of-stoplights-near-gas-stations-increases-gas-prices.}}

\hypertarget{observing-the-histograms-gas-stations-at-stoplights-have-a-higher-median-gas}{%
\section{Observing the histograms, gas stations at stoplights have a
higher median
gas}\label{observing-the-histograms-gas-stations-at-stoplights-have-a-higher-median-gas}}

\hypertarget{price-relative-to-gas-stations-without-stoplights.-the-median-price-of-gasoline}{%
\section{price relative to gas stations without stoplights. The median
price of
gasoline}\label{price-relative-to-gas-stations-without-stoplights.-the-median-price-of-gasoline}}

\hypertarget{gasoline-at-gas-stations-without-a-stoplight-is-1.80gal-while-the-median-price}{%
\section{gasoline at gas stations without a stoplight is \$1.80/gal,
while the median
price}\label{gasoline-at-gas-stations-without-a-stoplight-is-1.80gal-while-the-median-price}}

\hypertarget{of-gasoline-at-gas-stations-wiht-a-stoplight-is-1.90gal.-additionally-it-should}{%
\section{of gasoline at gas stations wiht a stoplight is \$1.90/gal.
Additionally, it
should}\label{of-gasoline-at-gas-stations-wiht-a-stoplight-is-1.90gal.-additionally-it-should}}

\hypertarget{be-noted-that-there-is-an-outlier-of-2.10gal-for-gas-stations-without-a-stoplight.}{%
\section{be noted that there is an outlier of \$2.10/gal for gas
stations without a
stoplight.}\label{be-noted-that-there-is-an-outlier-of-2.10gal-for-gas-stations-without-a-stoplight.}}

\#1E ggplot(data=gas\_prices) + geom\_boxplot(aes(x=Highway, y=Price,
fill = Highway))

\hypertarget{i-created-boxplots-to-assist-in-determining-if-a-gas-station-has-higher-gas-prices}{%
\section{I created boxplots to assist in determining if a gas station
has higher gas
prices}\label{i-created-boxplots-to-assist-in-determining-if-a-gas-station-has-higher-gas-prices}}

\hypertarget{as-a-consequence-of-direct-highway-access.-upon-observing-the-boxplots-it}{%
\section{as a consequence of direct highway access. Upon observing the
boxplots,
it}\label{as-a-consequence-of-direct-highway-access.-upon-observing-the-boxplots-it}}

\hypertarget{becomes-clear-that-gas-stations-with-direct-highway-access-increases-have-higher}{%
\section{becomes clear that gas stations with direct highway access
increases have
higher}\label{becomes-clear-that-gas-stations-with-direct-highway-access-increases-have-higher}}

\hypertarget{gas-prices.-gas-stations-with-direct-highway-access-have-a-higher-median-price-of}{%
\section{gas prices. Gas stations with direct highway access have a
higher median price
of}\label{gas-prices.-gas-stations-with-direct-highway-access-have-a-higher-median-price-of}}

\hypertarget{approximately-1.89gal-while-in-contrast-gas-statiosn-without-a-direct-highway}{%
\section{approximately \$1.89/gal, while in contrast, gas statiosn
without a direct
highway}\label{approximately-1.89gal-while-in-contrast-gas-statiosn-without-a-direct-highway}}

\hypertarget{access-have-a-lower-median-price-of-approximately-1.84gal.}{%
\section{access have a lower median price of approximately
\$1.84/gal.}\label{access-have-a-lower-median-price-of-approximately-1.84gal.}}

\#2 rides =
read.csv(`/Users/franklinstudent/Desktop/GitHub/ECO395M/data/bikeshare.csv')

\#Plot A ride\_total = rides \%\textgreater\% group\_by(hr)
\%\textgreater\% summarize(ride\_totals = sum(total))

ggplot(ride\_total) + geom\_line(aes(x = hr, y = ride\_totals))

\hypertarget{this-line-graph-displays-the-change-in-average-ridership-throughout-the-day.}{%
\section{This line graph displays the change in average ridership
throughout the
day.}\label{this-line-graph-displays-the-change-in-average-ridership-throughout-the-day.}}

\hypertarget{the-time-of-day-lays-on-the-x-axis-and-the-ridership-totals-lies-on-the-y-axis.}{%
\section{The time of day lays on the x-axis, and the ridership totals
lies on the
y-axis.}\label{the-time-of-day-lays-on-the-x-axis-and-the-ridership-totals-lies-on-the-y-axis.}}

\hypertarget{the-graph-shows-that-the-busiest-time-of-day-for-ridership-is-during-peak-rushhour.}{%
\section{The graph shows that the busiest time of day for ridership is
during peak
rushhour.}\label{the-graph-shows-that-the-busiest-time-of-day-for-ridership-is-during-peak-rushhour.}}

\#Plot B working\_day = rides \%\textgreater\% filter(workingday == `1')

non\_working\_day = rides \%\textgreater\% filter(workingday == `0')

working\_day\_total = working\_day \%\textgreater\% group\_by(hr)
\%\textgreater\% summarise(ride\_ = sum(total))

non\_working\_day\_total = non\_working\_day \%\textgreater\%
group\_by(hr) \%\textgreater\% summarise(ride\_ = sum(total))

day\_list = c(`1', `0')

combined\_rides = rides \%\textgreater\% filter(workingday \%in\%
day\_list) \%\textgreater\% group\_by(hr, workingday) \%\textgreater\%
summarize(total\_rides = sum(total))

ggplot(combined\_rides) + geom\_line(aes(x = hr, y = total\_rides, color
= workingday)) + facet\_wrap(\textasciitilde workingday)

\hypertarget{this-line-faceted-line-graph-was-made-to-help-determine-the-differences-in-ridership}{%
\section{This line faceted line graph was made to help determine the
differences in
ridership}\label{this-line-faceted-line-graph-was-made-to-help-determine-the-differences-in-ridership}}

\hypertarget{between-workdays-which-have-a-value-of-1-and-non-workdays-which-have-a-value-of-0.}{%
\section{between workdays, which have a value of 1, and non-workdays,
which have a value of
0.}\label{between-workdays-which-have-a-value-of-1-and-non-workdays-which-have-a-value-of-0.}}

\hypertarget{upon-observing-either-graph-its-easy-to-understand-that-the-highest-peaks-of-ridership}{%
\section{Upon observing either graph, its easy to understand that the
highest peaks of
ridership}\label{upon-observing-either-graph-its-easy-to-understand-that-the-highest-peaks-of-ridership}}

\hypertarget{occurs-during-a-workday-and-at-rushhour-times.-non-workdays-have-a-peak-during}{%
\section{occurs during a workday, and at rushhour times. Non-workdays
have a peak
during}\label{occurs-during-a-workday-and-at-rushhour-times.-non-workdays-have-a-peak-during}}

\hypertarget{mid-afternoons.}{%
\section{mid-afternoons.}\label{mid-afternoons.}}

\#Plot C rides = rides \%\textgreater\% filter(hr == `8')
\%\textgreater\% mutate(day = ifelse(workingday == 1, `Work Day',
``Non-Work Day''))

d1 = rides \%\textgreater\% group\_by(day, weathersit) \%\textgreater\%
summarize(average\_rides = mean(total))

ggplot(data = d1) + geom\_col(mapping = aes(x = weathersit, y =
average\_rides, fill = weathersit), position = `dodge') +
facet\_wrap(\textasciitilde day)

\hypertarget{in-addition-to-this-faceted-bar-graph-breaking-down-between-non-workdays-and}{%
\section{In addition to this faceted bar graph breaking down between
non-workdays
and}\label{in-addition-to-this-faceted-bar-graph-breaking-down-between-non-workdays-and}}

\hypertarget{workdays-differences-of-average-ridership-can-also-observed-in-reference-to}{%
\section{workdays, differences of average ridership can also observed in
reference
to}\label{workdays-differences-of-average-ridership-can-also-observed-in-reference-to}}

\hypertarget{weather-conditions.-at-first-glace-its-very-clear-that-average-ridership-is}{%
\section{weather conditions. At first glace, its very clear that average
ridership
is}\label{weather-conditions.-at-first-glace-its-very-clear-that-average-ridership-is}}

\hypertarget{much-higher-during-a-workday.-in-all-cases-ridership-decreases-as-the-weather}{%
\section{much higher during a workday. In all cases, ridership decreases
as the
weather}\label{much-higher-during-a-workday.-in-all-cases-ridership-decreases-as-the-weather}}

\hypertarget{worsens.}{%
\section{worsens.}\label{worsens.}}

\#3

abia =
read.csv(`/Users/franklinstudent/Desktop/GitHub/ECO395M/data/ABIA.csv')

airlines = c(`AA', `UA', `WN', `CO')

combined\_airlines = abia \%\textgreater\% filter(UniqueCarrier \%in\%
airlines) \%\textgreater\% group\_by(UniqueCarrier, DayOfWeek)
\%\textgreater\% summarise(total\_count = n(), cancelled = sum(Cancelled
== 1), delayed\_percentage = cancelled/total\_count)

ggplot(combined\_airlines) + geom\_line(aes(x = DayOfWeek, y =
total\_count, color = UniqueCarrier)) + scale\_x\_continuous(breaks =
1:7) + scale\_y\_log10()

\hypertarget{i-chose-4-airlines-to-observe-and-compare-throughout-the-week.-aa-american}{%
\section{I chose 4 airlines to observe and compare throughout the week.
AA
(American}\label{i-chose-4-airlines-to-observe-and-compare-throughout-the-week.-aa-american}}

\hypertarget{airlines-co-continental-airlines-ua-united-airlines-and-wn-southwest}{%
\section{Airlines), CO (Continental Airlines), UA (United Airlines), and
WN
(Southwest}\label{airlines-co-continental-airlines-ua-united-airlines-and-wn-southwest}}

\hypertarget{airlines.-wn-has-the-most-flights-at-aria-while-ua-has-by-far-the-lowest}{%
\section{Airlines). WN has the most flights at ARIA, while UA has, by
far, the
lowest}\label{airlines.-wn-has-the-most-flights-at-aria-while-ua-has-by-far-the-lowest}}

\hypertarget{number-of-flights.-all-airlines-have-a-dip-in-traffic-on-saturdays-before}{%
\section{number of flights. All airlines have a dip in traffic on
Saturdays,
before}\label{number-of-flights.-all-airlines-have-a-dip-in-traffic-on-saturdays-before}}

\hypertarget{increasing-again-for-sunday.-this-makes-intutive-sense-because-people-tend-to}{%
\section{increasing again for Sunday. This makes intutive sense because
people tend
to}\label{increasing-again-for-sunday.-this-makes-intutive-sense-because-people-tend-to}}

\hypertarget{fly-someplace-for-the-weekend-and-return-on-sunday-for-work-on-monday.}{%
\section{fly someplace for the weekend, and return on Sunday for work on
Monday.}\label{fly-someplace-for-the-weekend-and-return-on-sunday-for-work-on-monday.}}

ggplot(data = combined\_airlines) + geom\_col(mapping = aes(x =
UniqueCarrier, y = delayed\_percentage, fill = UniqueCarrier), position
= `dodge') + facet\_wrap(\textasciitilde DayOfWeek)

\hypertarget{i-then-decided-to-create-a-bar-graph-to-display-the-percentages-of-canceled}{%
\section{I then decided to create a bar graph to display the percentages
of
canceled}\label{i-then-decided-to-create-a-bar-graph-to-display-the-percentages-of-canceled}}

\hypertarget{flights-each-day-of-the-week-for-the-the-4-airlines-listed.-one-conclusion-made}{%
\section{flights each day of the week for the the 4 airlines listed. One
conclusion
made}\label{flights-each-day-of-the-week-for-the-the-4-airlines-listed.-one-conclusion-made}}

\hypertarget{is-that-although-wn-has-the-most-flights-out-of-abia-aa-has-the-highest}{%
\section{is that although WN has the most flights out of ABIA, AA has
the
highest}\label{is-that-although-wn-has-the-most-flights-out-of-abia-aa-has-the-highest}}

\hypertarget{percentages-of-cancelled-flights-on-nearly-every-day-of-the-week-with-tuesday}{%
\section{percentages of cancelled flights on nearly every day of the
week, with
Tuesday}\label{percentages-of-cancelled-flights-on-nearly-every-day-of-the-week-with-tuesday}}

\hypertarget{being-the-worst-day-to-fly-on-aa-but-the-best-day-to-fly-ua.}{%
\section{being the worst day to fly on AA, but the best day to fly
UA.}\label{being-the-worst-day-to-fly-on-aa-but-the-best-day-to-fly-ua.}}

\#4 sclass =
read.csv(`/Users/franklinstudent/Desktop/GitHub/ECO395M/data/sclass.csv')

class350 = sclass \%\textgreater\% filter(trim == `350')

class63 = sclass \%\textgreater\% filter(trim == `63 AMG')

ggplot(data = class350) + geom\_point(mapping = aes(x = mileage, y =
price), color = `darkgrey')

ggplot(data = class63) + geom\_point(mapping = aes(x = mileage, y =
price), color = `darkgrey')

class350\_split = initial\_split(class350, prop = 0.8) class350\_train =
training(class350\_split) class350\_test = testing(class350\_split)

class63\_split = initial\_split(class63, prop = 0.8) class63\_train =
training(class63\_split) class63\_test = testing(class63\_split)

lm1 = lm(price \textasciitilde{} mileage, data = class350\_train) lm2 =
lm(price \textasciitilde{} poly(mileage, 2), data = class350\_train)

lm3 = lm(price \textasciitilde{} mileage, data = class63\_train) lm4 =
lm(price \textasciitilde{} poly(mileage, 2), data = class63\_train)

knn350 = knnreg(price \textasciitilde{} mileage, data = class350\_train,
k = 30) rmse(knn350, class350\_test)

knn63 = knnreg(price \textasciitilde{} mileage, data = class63\_train, k
= 150) rmse(knn63, class63\_test)

class350\_test = class350\_test \%\textgreater\% mutate(Price\_pred =
predict(knn350, class350\_test))

class63\_test = class63\_test \%\textgreater\% mutate(Price\_pred =
predict(knn63, class63\_test))

p\_test350 = ggplot(data = class350\_test) + geom\_point(mapping = aes(x
= mileage, y = price), alpha=0.2) p\_test350

p\_test63 = ggplot(data = class63\_test) + geom\_point(mapping = aes(x =
mileage, y = price), alpha=0.2) p\_test63

p\_test350 + geom\_line(aes(x = mileage, y = Price\_pred), color =
`red', size = 1.5)

p\_test63 + geom\_line(aes(x = mileage, y = Price\_pred), color = `red',
size = 1.5)

rmse\_out350 = foreach(i=1:10, .combine=`c') \%do\% \{ class350\_split =
initial\_split(class350, prop = 0.8) class350\_train =
training(class350\_split) class350\_test = testing(class350\_split)

knn\_model = knnreg(price \textasciitilde{} mileage, data =
class350\_train, k = 25) modelr::rmse(knn\_model350, class350\_test) \}

rmse\_out350

rmse\_out63 = foreach(i=1:10, .combine=`c') \%do\% \{ class63\_split =
initial\_split(class63, prop = 0.8) class63\_train =
training(class63\_split) class63\_test = testing(class63\_split)

knn\_model63 = knnreg(price \textasciitilde{} mileage, data =
class63\_train, k = 25) modelr::rmse(knn\_model63, class63\_test) \}

rmse\_out63

\hypertarget{i-believe-that-the-trim-63-amg-has-a-higher-optimal-value-of-k-because-theres}{%
\section{I believe that the trim 63 AMG has a higher optimal value of k
because
theres}\label{i-believe-that-the-trim-63-amg-has-a-higher-optimal-value-of-k-because-theres}}

\hypertarget{a-larger-concentration-of-cars-valued-much-higher-with-mileage-close-to-zero}{%
\section{a larger concentration of cars valued much higher with mileage
close to
zero,}\label{a-larger-concentration-of-cars-valued-much-higher-with-mileage-close-to-zero}}

\hypertarget{whereas-the-350-trim-is-much-more-spread-out.}{%
\section{whereas the 350 trim is much more spread
out.}\label{whereas-the-350-trim-is-much-more-spread-out.}}

\end{document}
